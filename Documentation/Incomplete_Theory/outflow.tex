\documentclass[11pt,a4paper]{article}
\usepackage[utf8x]{inputenc}
%\usepackage{isolatin1}
\usepackage{ctable}
\usepackage[french]{babel}

%\usepackage[T1]{inputenc}
%\usepackage{float}
%\usepackage{vmargin}
\usepackage{epsfig}
\usepackage{graphicx}
\usepackage{graphics}

\setlength{\parindent}{10pt}
\setlength{\parskip}{10pt}
%\renewcommand{\baselinestretch}{2.0}
%\renewcommand\baselinestretch{1.5}
\newcommand\numero{{n$^{\rm o}\,\,$}}
%
%
%2345678901234567890123456789012345678901234567890123456789012345678901234567890


% (Jie Li?:) Ici, on definit trois macros

% la premiere est une macro avec deux parametres: le premier specifie
%       la largeur, le second le nom du fichier en EncasulatedPostscript ou se
%       trouve ton dessin.

\def\epsfx#1#2{\leavevmode\epsfxsize=#1 \epsfbox{#2}}

% la seconde est une macro avec deux parameters: mais le premier specifie
%       la hauteur, le second le nom en EncasulatedPostscript ou se
%       trouve ton dessin.

\def\epsfy#1#2{\leavevmode\epsfysize=#1 \epsfbox{#2}}

% la troisieme est une macro avec trois parameters: le premier specifie
%       la larheur, le second la hauteur, le troisieme le nom de fichier
%       en EncasulatedPostscript ou se trouve ton dessin

\def\epsfxy#1#2#3{\leavevmode\epsfxsize=#1 \epsfysize=#2 \epsfbox{#3}}


\newcommand\lgawebsite{{\tt ftp://ftp.jussieu.fr/jussieu/labos/lmm}}
% For Notes sections
\newcommand\subn{\paragraph{}}

\newcommand{\refeq}[1]{(\ref{#1})}
\newcommand{\reff}[1]{(\ref{#1})}

\newcommand{\trema}{\"}
%boldface letters

\newcommand\A{{\bf a}}
\newcommand\B{{\bf b}}
\newcommand\C{{\bf c}}
\newcommand\D{{\bf d}}
\newcommand\E{{\bf e}}
\newcommand\I{{\bf i}}
\newcommand\J{{\bf j}}
\newcommand\K{{\bf k}}
\newcommand\G{{\bf g}}
\newcommand\M{{\bf m}}
\newcommand\N{{\bf n}}
\renewcommand\P{{\bf p}}
\newcommand\Q{{\bf q}}
\newcommand\R{{\bf r}}
\newcommand\F{{\bf f}}
\newcommand\T{{\bf t}}
\newcommand\U{{\bf u}}
\newcommand\ut{{\tilde u}}
\newcommand\vt{{\tilde v}}
\newcommand\wt{{\tilde w}}
\newcommand\V{{\bf v}}
\newcommand\W{{\bf w}}
\newcommand\X{{\bf x}}
\newcommand\Y{{\bf y}}
\newcommand\Z{{\bf z}}

\renewcommand\AA{{\bf A}}
\newcommand\BB{{\bf B}}
\newcommand\CC{{\bf C}}
\newcommand\DD{{\bf D}}
\newcommand\EE{{\bf E}}
\newcommand\FF{{\bf F}}
\newcommand\MM{{\bf M}}
%\newcommand\NN{{\bf N}}
%\renewcommand\NN{{\typeout{Warning ! } \bf WARNING NN NOT AVAIL}}
\newcommand\II{{\bf I}}
\newcommand\JJ{{\bf J}}
\newcommand\KK{{\bf K}}
\newcommand\GG{{\bf G}}
\newcommand\PP{{\bf P}}
\newcommand\RR{{\bf R}}
\renewcommand\SS{{\bf S}}
\newcommand\TT{{\bf T}}
\newcommand\UU{{\bf U}}
\newcommand\UT{{\bf \tilde u}}
\newcommand\XX{{\bf X}}
\newcommand\YY{{\bf Y}}
\newcommand\ZZ{{\bf Z}}

\newcommand\MMM{{\cal M}}
\newcommand\NNN{{\cal N}}

\newcommand\derta{\partial_{t_1}}
\newcommand\dertb{\partial_{t_2}}
\newcommand\dert{\partial_t}
\newcommand\derx{\partial_x}
\newcommand\dery{\partial_y}
\newcommand\derz{\partial_z}
\newcommand\derr{\partial_r}
\newcommand\derth[1]{\frac{\partial #1}{\partial \theta}}
\newcommand\dera{\partial_\alpha}
\newcommand\derb{\partial_\beta}
\newcommand\deri[1]{ \frac{\partial{ #1}}{\partial{x_i}} }
\newcommand\derj[1]{ \frac{\partial{ #1}}{\partial{x_j}} }
\newcommand\derxj{ \frac{\partial}{\partial{x_j}} }
\newcommand\derxi{ \frac{\partial}{\partial{x_i}} }
\newcommand\dertt[1]{ \frac{\partial{ #1}}{\partial t} }
\newcommand\derft{ \frac{\partial}{\partial t} }
\newcommand\dertj[1]{\frac{\partial^2 #1}{\partial x_j^2}}
\newcommand\jac[2]{\frac{\partial(#1,#2)}{\partial(x,z)}}

\newcommand\derd[1]{\frac{D #1}{Dt}}
\newcommand\romandt[1]{\frac{{\rm d} #1}{{\rm d}t}}

\newcommand\der[2]{\frac{\partial #1}{\partial #2}}
\newcommand\pori[2]{\frac{\partial #2}{\partial #1}}
\newcommand\derbrack[2]{\frac{\partial}{\partial #2}\left[ #1 \right]}

\newcommand\romander[2]{\frac{{\rm d} #1}{{\rm d}#2}}

\newcommand\grad{\nabla}

% Greek

\newcommand\eps{\epsilon}
\newcommand\ii{{\rm i}}
\newcommand\om{{\omega}}

\newcommand\porh[1]{\frac{\partial H ( {\cal S} )}{\partial #1}}

\newcommand\be{\begin{equation}}
\newcommand\nd{\end{equation}}
\newcommand\bed{\begin{displaymath}}
\newcommand\ndd{\end{displaymath}}
\newcommand\hb[1]{\hbox{\hskip 10 pt #1 \hskip 10pt} }
\newcommand\ba{\begin{array}}
\newcommand\ea{\end{array}}

% Miscellaneous

\newcommand\Order{{\cal O}}

%\renewcommand\Re{{\rm Re}\,}
%\newcommand\re{{\rm Re}\,}
\newcommand\We{{\rm We}\,}
\newcommand\Ra{{\rm Ra}\,}
\newcommand\Ma{{\rm Ma}\,}
\newcommand\Bo{{\rm Bo}\,}
\newcommand\Ca{{\rm Ca}\,}
\newcommand\Oh{{\rm Oh}\,}
\newcommand\La{{\rm La}\,}
\newcommand\Nu{{\rm Nu}\,}
\newcommand\At{{\rm At}\,}



\begin{document}
\titlepage
\mbox{}
\begin{center}
\vskip 1cm
{\Large \bf Outflow
}\\\vskip 1cm
Stéphane Zaleski\\
\vskip 0.5cm
\today
\end{center}

Following the Book
$$
\partial u_n / \partial n = 0 
$$ 
Together with the incompressibility equation this implies
$$
\int_{S_{out}} u . n ds = \int_{S_{in}} u . n ds 
$$
on the exit surface $S_{out}$ but this is not satisfied by the ghost velocity on the exit face. 
If on the entry face u . n = U_in then in the projection method
one must set $\partial p / \partial n = 0$ on S_in, then 
$$
\int div (1/rho) grad p dv = \oint  (1/rho) grad p . ds
$$
and thus 
$$
\int div (1/rho) grad p dv = \int_S_out  (1/rho) grad p . n ds
$$
on the exit surface S.
Consider the simplest setup, akin to standard "wall" conditions, where
u=u_in is fixed on S_in and u=u_out is fixed on S_out with u_in S_in =
u_out S_out

On the exit face
a) either pressure is set at ghost points: p(ie+1)=P_out and then the
pressure equation at p(ie) is solved to obtain incompressibility at
index ie. Is u(ie+1/2) is computed ? recall index ieu = ie-1 if bdry_cond
 not periodic so ie is indeed not computed. But u(ie+1/2) could be extrapolated.

b) or u(ie+1/2) is set , for instance by extrapolation +
``correction of the mass flux'' (the meaning of ``correction of the mass
flux is given below''), and then u(ie+1/2) is not pressure corrected so
p(ie+1) = p(ie) and A_East = 0 and no pressure value is set. 

In case b), since p(ie+1) = p(ie) , then $\partial p / \partial n = 0$ and 
$$
\int div (1/rho) grad p dv = \int_S_out (1/rho) grad p . n ds = 0 
$$
so the solvabity condition is 
$$
\oint div (1/rho) grad p dv = \int div u^* dv = 0 
$$
which is the standard case which we know works. The solvability condition  is
ensured by correction of the mass flux for u^*. The correction of the mass flux is non-zero 
because div u^* is in general not zero (otherwise the pressure correction method
would not be necessary) . ``Correction of the mass flux'' means that an average outflow
velocity is computed
$$
\bar u_in = = (1/S_in) \int u_in ds
\bar u^*_out = = (1/S_out) \int u^*_out ds
$$
and after correction
$$
u^**_S_out = u^*_S_out - \bar u^*_out + \bar u^*_in
$$
In discrete terms, S_out is the surface ie+1/2 and u^*(ie+1/2) is first computed by 
extrapolation at first order
$$
 u^*(ie+1/2) =  u^*(ie-1/2) 
$$
In case a) let u^*(ie+1/2) be extrapolated but not corrected for the mass flux. 
The pressure correction is applied  to ensure incompressibility. The Poisson equation
has boundary condition p=P_out on S_out and $\partial u_n / \partial n = 0$
on S_in. Then we need proof that this is well posed. 

Attempt case a) as BC 5 and case b) as BC 4. 

\end{document}
