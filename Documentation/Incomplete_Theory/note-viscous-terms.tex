\documentclass[11pt,a4paper]{article}
\usepackage[utf8x]{inputenc}
\usepackage{isolatin1}
\usepackage{ctable}
\usepackage[french]{babel}

%\usepackage[T1]{inputenc}
%\usepackage{float}
%\usepackage{vmargin}
\usepackage{epsfig}
\usepackage{graphicx}
\usepackage{graphics}

\setlength{\parindent}{10pt}
\setlength{\parskip}{10pt}
%\renewcommand{\baselinestretch}{2.0}
%\renewcommand\baselinestretch{1.5}
\newcommand\numero{{n$^{\rm o}\,\,$}}
%
%
%2345678901234567890123456789012345678901234567890123456789012345678901234567890


% (Jie Li?:) Ici, on definit trois macros

% la premiere est une macro avec deux parametres: le premier specifie
%       la largeur, le second le nom du fichier en EncasulatedPostscript ou se
%       trouve ton dessin.

\def\epsfx#1#2{\leavevmode\epsfxsize=#1 \epsfbox{#2}}

% la seconde est une macro avec deux parameters: mais le premier specifie
%       la hauteur, le second le nom en EncasulatedPostscript ou se
%       trouve ton dessin.

\def\epsfy#1#2{\leavevmode\epsfysize=#1 \epsfbox{#2}}

% la troisieme est une macro avec trois parameters: le premier specifie
%       la larheur, le second la hauteur, le troisieme le nom de fichier
%       en EncasulatedPostscript ou se trouve ton dessin

\def\epsfxy#1#2#3{\leavevmode\epsfxsize=#1 \epsfysize=#2 \epsfbox{#3}}


\newcommand\lgawebsite{{\tt ftp://ftp.jussieu.fr/jussieu/labos/lmm}}
% For Notes sections
\newcommand\subn{\paragraph{}}

\newcommand{\refeq}[1]{(\ref{#1})}
\newcommand{\reff}[1]{(\ref{#1})}

\newcommand{\trema}{\"}
%boldface letters

\newcommand\A{{\bf a}}
\newcommand\B{{\bf b}}
\newcommand\C{{\bf c}}
\newcommand\D{{\bf d}}
\newcommand\E{{\bf e}}
\newcommand\I{{\bf i}}
\newcommand\J{{\bf j}}
\newcommand\K{{\bf k}}
\newcommand\G{{\bf g}}
\newcommand\M{{\bf m}}
\newcommand\N{{\bf n}}
\renewcommand\P{{\bf p}}
\newcommand\Q{{\bf q}}
\newcommand\R{{\bf r}}
\newcommand\F{{\bf f}}
\newcommand\T{{\bf t}}
\newcommand\U{{\bf u}}
\newcommand\ut{{\tilde u}}
\newcommand\vt{{\tilde v}}
\newcommand\wt{{\tilde w}}
\newcommand\V{{\bf v}}
\newcommand\W{{\bf w}}
\newcommand\X{{\bf x}}
\newcommand\Y{{\bf y}}
\newcommand\Z{{\bf z}}

\renewcommand\AA{{\bf A}}
\newcommand\BB{{\bf B}}
\newcommand\CC{{\bf C}}
\newcommand\DD{{\bf D}}
\newcommand\EE{{\bf E}}
\newcommand\FF{{\bf F}}
\newcommand\MM{{\bf M}}
%\newcommand\NN{{\bf N}}
%\renewcommand\NN{{\typeout{Warning ! } \bf WARNING NN NOT AVAIL}}
\newcommand\II{{\bf I}}
\newcommand\JJ{{\bf J}}
\newcommand\KK{{\bf K}}
\newcommand\GG{{\bf G}}
\newcommand\PP{{\bf P}}
\newcommand\RR{{\bf R}}
\renewcommand\SS{{\bf S}}
\newcommand\TT{{\bf T}}
\newcommand\UU{{\bf U}}
\newcommand\UT{{\bf \tilde u}}
\newcommand\XX{{\bf X}}
\newcommand\YY{{\bf Y}}
\newcommand\ZZ{{\bf Z}}

\newcommand\MMM{{\cal M}}
\newcommand\NNN{{\cal N}}

\newcommand\derta{\partial_{t_1}}
\newcommand\dertb{\partial_{t_2}}
\newcommand\dert{\partial_t}
\newcommand\derx{\partial_x}
\newcommand\dery{\partial_y}
\newcommand\derz{\partial_z}
\newcommand\derr{\partial_r}
\newcommand\derth[1]{\frac{\partial #1}{\partial \theta}}
\newcommand\dera{\partial_\alpha}
\newcommand\derb{\partial_\beta}
\newcommand\deri[1]{ \frac{\partial{ #1}}{\partial{x_i}} }
\newcommand\derj[1]{ \frac{\partial{ #1}}{\partial{x_j}} }
\newcommand\derxj{ \frac{\partial}{\partial{x_j}} }
\newcommand\derxi{ \frac{\partial}{\partial{x_i}} }
\newcommand\dertt[1]{ \frac{\partial{ #1}}{\partial t} }
\newcommand\derft{ \frac{\partial}{\partial t} }
\newcommand\dertj[1]{\frac{\partial^2 #1}{\partial x_j^2}}
\newcommand\jac[2]{\frac{\partial(#1,#2)}{\partial(x,z)}}

\newcommand\derd[1]{\frac{D #1}{Dt}}
\newcommand\romandt[1]{\frac{{\rm d} #1}{{\rm d}t}}

\newcommand\der[2]{\frac{\partial #1}{\partial #2}}
\newcommand\pori[2]{\frac{\partial #2}{\partial #1}}
\newcommand\derbrack[2]{\frac{\partial}{\partial #2}\left[ #1 \right]}

\newcommand\romander[2]{\frac{{\rm d} #1}{{\rm d}#2}}

\newcommand\grad{\nabla}

% Greek

\newcommand\eps{\epsilon}
\newcommand\ii{{\rm i}}
\newcommand\om{{\omega}}

\newcommand\porh[1]{\frac{\partial H ( {\cal S} )}{\partial #1}}

\newcommand\be{\begin{equation}}
\newcommand\nd{\end{equation}}
\newcommand\bed{\begin{displaymath}}
\newcommand\ndd{\end{displaymath}}
\newcommand\hb[1]{\hbox{\hskip 10 pt #1 \hskip 10pt} }
\newcommand\ba{\begin{array}}
\newcommand\ea{\end{array}}

% Miscellaneous

\newcommand\Order{{\cal O}}

\renewcommand\Re{{\rm Re}\,}
%\newcommand\re{{\rm Re}\,}
\newcommand\We{{\rm We}\,}
\newcommand\Ra{{\rm Ra}\,}
\newcommand\Ma{{\rm Ma}\,}
\newcommand\Bo{{\rm Bo}\,}
\newcommand\Ca{{\rm Ca}\,}
\newcommand\Oh{{\rm Oh}\,}
\newcommand\La{{\rm La}\,}
\newcommand\Nu{{\rm Nu}\,}
\newcommand\At{{\rm At}\,}



\begin{document}
\titlepage
\mbox{}
Basic equations
\be
f(u) + s(u) = 0  \hb{and} \nabla \cdot u = 0. 
\nd
where in the Navier-Stokes case
\be
f(u) = \Delta u + g
\nd
and
\be
s(u) = - \nabla p 
\nd
where the pressure is whatever is necessary for incompressibility.
One step of Gauss-Seidel
for the Laplacian operator. 
\be
u^{n,*,k} = M(u^{n,k},g - \nabla p^{k-1}) 
\nd
where for any function $h$
\be
M(v,h) = v \hb{if}  \Delta v + h = 0.
\nd
The next iteration is defined by 
\be
u^{n,k+1} = u^{n,*,k} + \nabla \pi^{k} \hb{and} \nabla \cdot u^{n,k+1} = 0  
\nd
and
\be 
p^k = p^{k-1} + \pi^k.
\nd
At convergence
\be
u^{n,k+1} = u^{n,k}
\nd
and
\be
\pi^k = 0.
\nd
Then
\be
u^{n,k+1}  = M(u^{n,k},g - \nabla p^{k-1}) .
\nd
Let $v = u^{n,k}$ then
\be
v = M(v,g - \nabla p)
\nd
and thus
\be
\Delta v + g - \nabla p = 0 ,
\nd
achieving
\be
f(v) + s(v)  = 0 \hb{and} \nabla \cdot v = 0. 
\nd
\newpage

 
Since
\be
\nabla \cdot f(u) - \Delta p = 0
\nd
then
\be
p = \Delta^{-1} ( \nabla \cdot f(u) ) ,
\nd
(boundary conditions on $p$ are not specified at this point although
I believe that $n \cdot \nabla p = 0$ should work.)
Then if
\be
h(w) = - \nabla \Delta^{-1} ( \nabla \cdot w ) 
\nd
then
\be
s(u) = h[f(u)]
\nd 
with bc on solids. If we solve using a time evolution
\be
\dert u = f(u) + s(u) = 0  \hb{and} \nabla \cdot u = 0. 
\nd
then an explici method is
\be
u^* = u^n + \tau f(u^n)
\nd
and
\be
u^{n+1} = u^* + h(u^*) .
\nd
Implicit:
\be
u^* = u^n + \tau f(u^*)
\nd
and
\be
u^{n+1} = u^* + h(u^*).
\nd
If explicit and $u^{n+1} = u^n = v$ then
\be
v = v + \tau f(v) + \tau h[f(v)] 
\nd
and thus we solve the problem
\be
L(v) = f(v) + h[f(v)] = 0 \hb{and} \nabla \cdot v = 0 
\nd
If implicit and  $u^{n+1} = u^n = v$ then
\be
v = v + \tau f(u^*) + h(u^*) 
\nd
and
so we solve a different problem
\be
h(u^*) + f(u^*) =0 \hb{and} \nabla \cdot v = 0 
\nd
with $v$ related to $u^*$ as above. 

\newpage
\begin{center}
\vskip 1cm
{\Large \bf Note on viscous terms
}\\\vskip 1cm
Stéphane Zaleski\\
\vskip 0.5cm
{\small  \it
D'ALEMBERT\\
} 
\vskip 0.5 cm 
{\it email: zaleski@dalembert.upmc.fr}
\vskip 0.5 cm
\today
\end{center}

\newpage
The viscous term is
\be
\F_v =\nabla \cdot \{ \mu [\nabla\U  + (\nabla\U)^T] \}
= \nabla \mu \cdot [\nabla\U  + (\nabla\U)^T]  + \mu \Delta\U
\nd
The term computed in {\sf MomentumDiffusion} is 
\be
\delta u = \frac 1 {\delta y}  \left( \mu_{i+1/2,j+1/2} \left.\frac {\delta v }{\delta y}\right\vert_{i+1/2,j+1/2} - \mu_{i+1/2,j-1/2} \left.\frac {\delta v }{\delta y}\right\vert_{i+1/2,j-1/2}\right)
\nd
using
$ac - bd =1/2[ (a-b)(c+d) + (a+b)(c-d) ] =1/2[ ac - bd + ad - bc + ac - bd - ad + bc]$

\begin{eqnarray}
\delta u & = & \frac 1 {2 \delta y} \left[  \left( \mu_{i+1/2,j+1/2}  + \mu_{i+1/2,j-1/2}
\right) \left( \left.\frac {\delta v }{\delta y}\right\vert_{i+1/2,j+1/2} - \left.\frac {\delta v }{\delta y}\right\vert_{i+1/2,j-1/2}\right)\right.\nonumber \\
& + &  \left.\left( \mu_{i+1/2,j+1/2}  - \mu_{i+1/2,j-1/2} \right) 
\left( \left.\frac {\delta v }{\delta y}\right\vert_{i+1/2,j+1/2} + \left.\frac {\delta v }{\delta y}\right\vert_{i+1/2,j-1/2}\right)\right]
\end{eqnarray}

\end{document}
